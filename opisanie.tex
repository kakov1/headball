\documentclass{article}
\usepackage[russian]{babel}
\usepackage{graphicx}
\usepackage{hyperref}

\title{Техническое описание проекта "HeadBall"}
\author{}
\date{}

\begin{document}

\maketitle

\section{Общая информация}
\textbf{Название проекта:} HeadBall  \\
\textbf{Тип проекта:} Сетевая 2D-игра  \\
\textbf{Технологии:} Python, Pygame, Pymunk, C (серверная часть)  \\
\textbf{Архитектура:} Клиент-серверная  

\section{Описание игры}
HeadBall — это многопользовательская аркадная игра в жанре футбола, в которой два игрока управляют персонажами, пытаясь забить мяч в ворота соперника.

\section{Компоненты проекта}
\subsection{Клиентская часть (Python, Pygame, Pymunk)}
\textbf{Основные файлы:}
\begin{itemize}
    \item \texttt{main.py} — главный игровой процесс, обработка ввода, графический интерфейс
    \item \texttt{params.py} — настройки игры (размеры экрана, параметры игроков и мяча, цвета, координаты начальных позиций)
\end{itemize}

\textbf{Функционал клиента:}
\begin{itemize}
    \item Отрисовка игрового поля и объектов (игроки, мяч, ворота)
    \item Обработка событий клавиатуры для управления персонажем
    \item Физический движок (Pymunk) для реалистичного поведения мяча и игроков
    \item Подключение к серверу, отправка данных о движении игрока и получение состояния второго игрока
\end{itemize}

\subsection{Серверная часть (C)}
\textbf{Основные файлы:}
\begin{itemize}
    \item \texttt{server.c} — обработка подключения игроков, передача сообщений
    \item \texttt{serverUtil.c} — вспомогательные функции для работы с сетью
    \item \texttt{serverUtil.h} — заголовочный файл с объявлениями функций
\end{itemize}

\textbf{Функционал сервера:}
\begin{itemize}
    \item Ожидание подключений от двух клиентов
    \item Прием и обработка данных от игроков
    \item Пересылка информации между клиентами (состояния управления)
    \item Обработка отключения игроков
\end{itemize}

\section{Логика игрового процесса}
\begin{enumerate}
    \item Сервер ожидает подключения двух игроков.
    \item Первый подключившийся клиент ждет второго игрока.
    \item После подключения второго игрока начинается игра.
    \item Игроки управляют персонажами с помощью клавиш \texttt{LEFT}, \texttt{RIGHT}, \texttt{UP}.
    \item Мяч и игроки взаимодействуют согласно физическим законам (Pymunk).
    \item Если мяч попадает в ворота, засчитывается гол, позиции игроков и мяча сбрасываются.
    \item Если один из игроков отключается, второй получает уведомление о выходе соперника.
\end{enumerate}

\section{Сетевая коммуникация}
\textbf{Протокол связи:} TCP  \\
\textbf{Обмен данными:}
\begin{itemize}
    \item Клиенты отправляют на сервер состояние управления (\texttt{000}, \texttt{100}, \texttt{010} и т. д.)
    \item Сервер пересылает эти данные другому клиенту
    \item Кодирование сообщений в текстовом формате
\end{itemize}
\end{document}


